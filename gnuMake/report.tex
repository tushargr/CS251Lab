\documentclass[a4paper, 10pt]{report}


\usepackage{lipsum,lineno}
\usepackage{graphicx}
\usepackage{subcaption}
\usepackage{hyperref}


\usepackage{geometry}
 \geometry{
 a4paper,
 total={170mm,257mm},
 left=22mm,
 top=22mm,
 }


\title{Report}
\author{Tushar Gurjar}
\date{}
\begin{document}
\maketitle
 


\begin{figure}
\centering
\includegraphics[width=\columnwidth]{scatter_1.eps}
 \caption{Scatter plot for Thread1}
 \label{fig:scatter_1}
\end{figure}
This is first scatter plot.Time is in microseconds.
This scatter plot is for Thread1.This plot is between number of elements on x-axis and corresponding time of execution on y-axis.
\newpage
\begin{figure}
\centering
\includegraphics[width=\columnwidth]{scatter_2.eps}
 \caption{Scatter plot for Thread2}
 \label{fig:scatter_2}
\end{figure}
This is the second scatter plot. Time is in microseconds.
This scatter plot is for Thread2.This plot is between number of elements on x-axis and corresponding time of execution on y-axis.
\newpage
\begin{figure}
\centering
\includegraphics[width=\columnwidth]{scatter_4.eps}
 \caption{Scatter plot for Thread4}
 \label{fig:scatter_4}
\end{figure}
This is the third scatter plot. Time is in microseconds.
This scatter plot is for Thread4.This plot is between number of elements on x-axis and corresponding time of execution on y-axis.
\newpage
\begin{figure}
\centering
\includegraphics[width=\columnwidth]{scatter_8.eps}
 \caption{Scatter plot for Thread8}
 \label{fig:scatter_8}
\end{figure}
This is the fourth scatter plot. Time is in microseconds.
This scatter plot is for Thread8.This plot is between number of elements on x-axis and corresponding time of execution on y-axis.
\newpage
\begin{figure}
\centering
\includegraphics[width=\columnwidth]{scatter_16.eps}
 \caption{Scatter plot for Thread16}
 \label{fig:scatter_16}
\end{figure}
This is the fifth scatter plot. Time is in microseconds.
This scatter plot is for Thread16.This plot is between number of elements on x-axis and corresponding time of execution on y-axis.






\newpage
\begin{figure}
\centering

\includegraphics[width=\columnwidth]{single_1.eps}
 \caption{Line plot for all threads}
 \label{fig:single_1}
\end{figure}
This is first line plot. Time is in microseconds.
This graph is a line plot for all the threads with number of elements on x-axis and average time for 100 sample on y-axis.

\newpage
\begin{figure}
\includegraphics[width=\columnwidth]{bar.eps}
 \caption{This is Bar\_Plot for all threads}
 \label{fig:bar}
\end{figure}
This is the first bar graph. Here bar1 in each element is 1 and other bars are with reference to it.
This plot is a bar graph for Threads with number of elements on x-axis and average speedup time on y-axis.Here time is in microoseconds.
\newpage
\begin{figure}
\includegraphics[width=\columnwidth]{ebar.eps}
 \caption{This is Error\_Bar\_Plot for all threads}
 \label{fig:error_bar}
\end{figure}
This is the first bar graph. Here bar1 in each element is 1 and other bars are with reference to it.
This plot is also a kind of bar plot with error-bars as additional feature.Here error bars have been calculated using variance.X-axis denotes number of elements and Y-axis denotes speedup time in microseconds. 



\end{document}

